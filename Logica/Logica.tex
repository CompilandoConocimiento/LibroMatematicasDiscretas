% ****************************************************************************************
% ************************     	LOGICA MATEMATICA   	  ********************************
% ****************************************************************************************


% =======================================================
% =======         HEADER FOR DOCUMENT        ============
% =======================================================
    % *********   DOCUMENT ITSELF   **************
    \documentclass[12pt]{report}                                    %Type of docuemtn and size of font
    \usepackage[margin=1.2in]{geometry}                             %Margins and Geometry pacakge
    \usepackage{ifthen}                                             %Allow simple programming
    \usepackage{hyperref}                                           %Create MetaData for a PDF and LINKS!
    \setlength{\parindent}{0pt}                                     %Eliminate ugly indentation
    \author{Oscar Andrés Rosas}                                     %Who I am

    % *********   LANGUAJE AND UFT-8   *********
    \usepackage[spanish]{babel}                                     %Please use spanish
    \usepackage[utf8]{inputenc}                                     %Please use spanish - UFT
    \usepackage[T1]{fontenc}                                        %Please use spanish

    % *********   MATH AND HIS STYLE  *********
    \usepackage{amsthm, amssymb, amsfonts, mathrsfs}                %Make math beautiful
    \usepackage[fleqn]{amsmath}                                     %Please make equations left
    \decimalpoint                                                   %Use decimal point

    % *********   GRAPHICS AND IMAGES *********
    \usepackage{graphicx}                                           %Allow to create graphics
    \usepackage{wrapfig}                                            %Allow to create images
    \graphicspath{ {Graphics/} }                                    %Where are the images :D

    % *********   LISTS AND TABLES ***********
    \usepackage{listings}                                           %We will be using code here
    \usepackage[inline]{enumitem}                                   %We will need to enumarate
    \usepackage{tasks}                                              %Horizontal lists
    \usepackage{longtable}                                          %Lets make tables awesome
    \usepackage{booktabs}                                           %Lets make tables awesome
    \usepackage{tabularx}                                           %Lets make tables awesome
    \usepackage{multirow}                                           %Lets make tables awesome
    \usepackage{multicol}                                           %Create multicolumns

    % *********   HEADERS AND FOOTERS ********
    \usepackage{fancyhdr}                                           %Lets make awesome headers/footers
    \pagestyle{fancy}                                               %Lets make awesome headers/footers
    \setlength{\headheight}{16pt}                                   %Top line
    \setlength{\parskip}{0.5em}                                     %Top line
    \renewcommand{\footrulewidth}{0.5pt}                            %Bottom line

    \lhead{                                                         %Left Header
        \hyperlink{chapter.\arabic{chapter}}                        %Make a link to the current chapter
        {\normalsize{\textsc{\nouppercase{\leftmark}}}}             %And fot it put the name
    }

    \rhead{                                                         %Right Header
        \hyperlink{section.\arabic{chapter}.\arabic{section}}       %Make a link to the current chapter
            {\footnotesize{\textsc{\nouppercase{\rightmark}}}}      %And fot it put the name
    }

    \rfoot{\textsc{\small{\hyperref[sec:Index]{Ve al Índice}}}}    %This will always be a footer  

    \fancyfoot[L]{                                                  %Algoritm for a changing footer
        \ifthenelse{\isodd{\value{page}}}                           %IF ODD PAGE:
            {\href{https://compilandoconocimiento.com/yo/}          %DO THIS:
                {\footnotesize                                      %Send the page
                    {\textsc{Oscar Andrés Rosas}}}}                 %Send the page
            {\href{https://compilandoconocimiento.com}              %ELSE DO THIS: 
                {\footnotesize                                      %Send the author
                    {\textsc{Compilando Conocimiento}}}}            %Send the author
    }
    
    
    
% ========================================
% ===========   COMMANDS    ==============
% ========================================

    % =====  GENERAL MATH  ==========
    \DeclareMathOperator \Space {\quad}                             %Use: \Space for a cool mega space
    \DeclareMathOperator \MiniSpace {\;}                            %Use: \Space for a cool mini space
    \newcommand \Such {\MiniSpace|\MiniSpace}                       %Use: \Such like in sets

    % =====  LOGIC  ==================
    \DeclareMathOperator \doublearrow {\leftrightarrow}             %Use: \doublearrow for a double arrow
    \newcommand \lequal {\MiniSpace \Leftrightarrow \MiniSpace}     %Use: \lequal for a double arrow
    \newcommand \linfire {\MiniSpace \Rightarrow \MiniSpace}        %Use: \lequal for a double arrow

    % =====  NUMBER THEORY  ==========
    \DeclareMathOperator \Naturals {\mathbb{N}}                     %Use: \Naturals por Notation
    \DeclareMathOperator \Integers {\mathbb{Z}}                     %Use: \Integers por Notation
    \DeclareMathOperator \Racionals{\mathbb{Q}}                     %Use: \Racionals por Notation
    \DeclareMathOperator \Reals {\mathbb{R}}                        %Use: \Reals por Notation
    \DeclareMathOperator \Complexs {\mathbb{C}}                     %Use: \Complex por Notation

    % === LINEAL ALGEBRA & VECTORS ===
    \DeclareMathOperator \LinealTransformation {\mathcal{T}}        %Use: \LinealTransformation for a cool T

    \newcommand{\pVector}[1]{                                       %Use: \pVector {Matrix Notation} use parentesis
        \ensuremath{\begin{pmatrix}#1\end{pmatrix}}                 %Example: \pVector{a\\b\\c} or \pVector{a&b&c} 
    }
    \newcommand{\lVector}[1]{                                       %Use: \lVector {Matrix Notation} use a abs 
        \ensuremath{\begin{vmatrix}#1\end{vmatrix}}                 %Example: \lVector{a\\b\\c} or \lVector{a&b&c} 
    }
    \newcommand{\Vector}[1]{                                        %Use: \Vector {Matrix Notation} no parentesis
        \ensuremath{\begin{matrix}#1\end{matrix}}                   %Example: \Vector{a\\b\\c} or \Vector{a&b&c}
    }



% =====================================================
% ============     	  COVER PAGE	   ================
% =====================================================
\begin{document}
\begin{titlepage}

	\center
	% ============ UNIVERSITY NAME AND DATA =========
	\textbf{\textsc{\Large Proyecto Compilando Conocimiento}}\\[1.0cm] 
	\textsc{\Large Matemáticas Discretas}\\[1.0cm] 

	% ============ NAME OF THE DOCUMENT  ============
	\rule{\linewidth}{0.5mm} \\[1.0cm]
		{ \huge \bfseries Lógica Matemática}\\[1.0cm] 
	\rule{\linewidth}{0.5mm} \\[2.0cm]
	
	% ====== SEMI TITLE ==========
	{\LARGE Una Pequeña Introducción}\\[7cm] 
	
	% ============  MY INFORMATION  =================
	\begin{center} \large
	\textbf{\textsc{Autor:}}\\
	Rosas Hernandez Oscar Andres
	\end{center}

	\vfill

\end{titlepage}

% =====================================================
% ========                INDICE              =========
% =====================================================
\tableofcontents{}
\label{sec:Index}

\clearpage










% ======================================================================================
% =============================       PRINCIPIOS BASICOS      ==========================
% ======================================================================================
\chapter{Proposiciones y Conectores}
    \clearpage

    % =====================================================
    % ============           DEFINICION            ========
    % =====================================================
    \section{Proposiciones}

        La lógica es una forma sistemática de pensar que nos permite deducir nueva información desde la
        información que ya conocemos.

        Recuerda que la lógica es un proceso de deducir la información correctamente,
        no sólo deducir la información correcta.

        La lógica trabajo con algo llamado proposiciones, son como las funciones para
        cálculo, o los lenguajes de programación para informática o los libros para la literatura.

        Así que empecemos por ahí ... ¿Qué son?


        % =====================================
        % =========   ¿QUE SON?     ===========
        % =====================================
        \subsection*{Definición}
            
            \textbf{Son proposiciones las frases que pueden adquirir un valor de verdadero o falso.}
            
            O dicho de manera formal:

            \textbf{Es una oración aseverativa de la que tiene sentido decir que es verdadera o falsa}.\\

            Y cuando digo frase, me refiero a:
            \begin{itemize}
                \item Secuencia finita de signos con significado y sentido de ser calificado como verdadero o falso.
                        (es decir una simple expresión matemática).

                \item Expresión lingüística susceptible de ser calificada de verdadera o falsa.
                        (es decir una frase aseverativa).
            \end{itemize}


        % =====================================
        % ====   SENTENCIAS ABIERTAS    =======
        % =====================================
        \subsection*{Sentencias Abiertas}
            Existen cosas que son parecidas a las proposiciones, pero no lo son exactamente, son cosas como:

            $p(x)$: $x$ es un número par.

            Puesto que la validez de $p(x)$ depende que número sea $x$, $p(x)$no es no totalmente cierta ni
            totalmente falsa, por lo tanto no es una proposición.

            Una oración como esta, cuya verdad depende del valor de una o más variables,
            se llama sentencias abierta.



            % ==========================
            % =====   EJEMPLOS   =======
            % ==========================
            \clearpage
            \subsection*{Ejemplo}


                Por ejemplo son proposiciones frases como:
                \begin{itemize}
                    \item $2 + 3 = 4$
                    \item Hay solamente 325 personas en Marte
                    \item $\forall x, y \in \Naturals$ se tiene que $\MiniSpace x+y \in \Reals$
                    \item Hoy es lunes
                    \item $f(x+y) = f(x) + f(y)$
                    \item Si x = 2 entonces 2x = 4
                \end{itemize}

                Pero no son cosas como:
                \begin{itemize}
                    \item ¡Ojalá no llueva hoy!
                    \item Haz la tarea
                    \item Este enunciado es falso
                    \item Tomar una siesta
                \end{itemize}



        % =====================================
        % ======   CLASIFICACION    ===========
        % =====================================
        \clearpage
        \subsection{Teoremas, Colorario y Tautológias}
            
            \subsubsection*{Clasificación de Propiedades}

                \begin{itemize}
                    \item \textbf{Tautología}: Cuando para todos los valores posibles de un conjunto
                            de proposiciones siempre será verdadero el conjunto.

                    \item \textbf{Contradicción}: Cuando para todos los valores posibles de un conjunto
                    de proposiciones esta será siempre falso.

                    \item \textbf{Contingencia}: Una proposición “común” son básicamente todas las que
                    no son ni tautologías ni contradicciones.
                \end{itemize}

            \subsubsection*{Notación}

                Además a los matemáticas les encanta demostrar todo y cuando digo todo, es TODO, así que
                aquí te dejo las diferencias entre varias palabras que se parecen:

                \begin{itemize}
                    \item \textbf{Proposición}: Enunciado que encierra un valor de verdad.

                    \item \textbf{Axioma}: Principio tan claro y evidente que no necesita demostración.

                    \item \textbf{Corolario}: Proposición demostrado que provoca una afirmación.

                    \item \textbf{Demostración}: Razonamiento por el cuál se da prueba de la
                        exactitud de una proposición.

                    \item \textbf{Lema}: Proposición que es necesaria demostrar antes de
                    establecer un teorema.

                \end{itemize}


    % =====================================================
    % ============           CONECTORES            ========
    % =====================================================
    \clearpage
    \section{Conectores Lógicos}

        Los conectores nos permiten 'concatenar' proposiciones o crear proposiciones mas avanzadas. 
        Veamos primero como solemos mostrarlos:

        \begin{longtable}{p{35mm} || p{30mm} || p{80mm}}

            % ==== HEADERS ============
            \textbf{\large Conector}
            &
            \textbf{\large Nombres}
            &
            \textbf{\large Símbolos}
            \\[1.5ex]
            \hline\hline
            & & \\                                                                    
            \endhead                                                     

            \large y  &  \large{p $\land$ q}                                    &

            \begin{minipage}[t]{\textwidth}\begin{itemize}
                \item \textbf{Conjunción de} p \textbf{y de} q
            \end{itemize}\end{minipage}                                                 \\[1.5ex]
            
            \hline & & \\ \large o  &  \large{p $\lor$ q}                               &
            
            \begin{minipage}[t]{\textwidth}\begin{itemize}
                \item \textbf{Disyunción de} p \textbf{y de} q
            \end{itemize}\end{minipage}                                                 \\[1.5ex]


            \hline & & \\ \large no  &  \large{$\lnot$ q}                               &

            \begin{minipage}[t]{\textwidth}\begin{itemize}
                \item \textbf{Negación de} P
            \end{itemize}\end{minipage}                                                 \\[1.5ex]

            \hline & & \\ \large implica  &  \large{p $\to$ q}                          & 

            \begin{minipage}[t]{\textwidth}\begin{itemize}
            \small{
                \item p \textbf{implica} q
                \item \textbf{Si} p\textbf{, entonces} q
                \item q \textbf{si} p
                \item \textbf{Sólo si} q \textbf{entonces} p
                \item p \textbf{sólo si} q
                \item \textbf{Cuando} p\textbf{,} q
                \item \textbf{Siempre que} q\textbf{,} p
                \item q \textbf{siempre que} p
                \item p \textbf{es una condición suficiente para} q
                \item q \textbf{es una condición necesaria para} p
                \item \textbf{Es necesario que} q \textbf{para} p
                \item \textbf{Es suficiente que} p \textbf{para que} q
            }\\
            \end{itemize}\end{minipage}                                                 \\[1.5ex]

            \hline & & \\ \large si y solo si  &  \large{p $\doublearrow$ q}            &

            \begin{minipage}[t]{\textwidth}\begin{itemize}
            \small{
                \item p \textbf{ssi} q
                \item p \textbf{es equivalente a} q
                \item p \textbf{es una condición necesaria y suficiente para} q
                \item \textbf{Para que} p \textbf{es necesario y suficiente que} q
            }\\
            \end{itemize}\end{minipage}                                                 \\
 
        \end{longtable}

        % =====================================
        % ======     NEGACION     ===========
        % =====================================
        \clearpage

        Las que siguen a continuación son lo que yo denomino más operaciones mas básicas en lógica.

        \subsection{Negación}

            Devuelve el inverso del valor de verdad de la proposición que le pases.\\

            \begin{tabular}{ |c|c|c| } 
                \hline &&\\
                \large{Nombre} & $p$ & $\lnot p$ \\[0.5em]
                \hline
                \multirow{2}{5em}{Negación}
                & $F$ & $V$  \\ \cline{2-3}
                & $V$ & $F$  \\ \cline{2-3}
                \hline
            \end{tabular}

        % =====================================
        % ======     CONJUNCION     ===========
        % =====================================
        \subsection{Conjunción}

            Devuelve verdadero \textbf{solo} cuando ambas son verdaderas, y falso en cualquier
            otra combinación.\\

            \begin{tabular}{ |c|c|c|c|c| } 
                \hline &&&\\
                \large{Nombre} & $p$ & $q$ & $p \land q$ \\[0.5em]
                \hline
                \multirow{4}{5em}{Conjunción}
                & $F$ & $F$ & $F$ \\ \cline{2-4}
                & $F$ & $V$ & $F$ \\ \cline{2-4}
                & $V$ & $F$ & $F$ \\ \cline{2-4}
                & $V$ & $V$ & $V$ \\ 
                \hline
            \end{tabular}


        % =====================================
        % ======     DISYUNCION     ===========
        % =====================================
        \subsection{Disyunción}

            Devuelve falso \textbf{solo} cuando ambas son falsas, y verdadero en cualquier
            otra combinación.\\

            \begin{tabular}{ |c|c|c|c|c| } 
                \hline &&&\\
                \large{Nombre} & $p$ & $q$ & $p \lor q$ \\[0.5em]
                \hline
                \multirow{4}{5em}{Disyunción}
                & $F$ & $F$ & $F$ \\ \cline{2-4}
                & $F$ & $V$ & $V$ \\ \cline{2-4}
                & $V$ & $F$ & $V$ \\ \cline{2-4}
                & $V$ & $V$ & $V$ \\ 
                \hline
            \end{tabular}





        % =====================================
        % ======     IMPLICACION    ===========
        % =====================================
        \clearpage
        \subsection{Implicación}

            Devuelve falso \textbf{solo} cuando la primera premisa es verdadera, pero la 
            segunda es falsa, y verdadero en cualquier otra combinación.

            Ve a $p \to q$ como una promesa de que siempre que $p$ es verdadera, $q$ será verdadera también.
            Sólo hay una manera de romper esta promesa y que es si $P$ sea verdad y $q$ es falso.\\


            \begin{tabular}{ |c|c|c|c|c| } 
                \hline &&&\\
                \large{Nombre} & $p$ & $q$ & $p \to q$ \\[0.5em]
                \hline
                \multirow{4}{5em}{Disyunción}
                & $F$ & $F$ & $V$ \\ \cline{2-4}
                & $F$ & $V$ & $V$ \\ \cline{2-4}
                & $V$ & $F$ & $F$ \\ \cline{2-4}
                & $V$ & $V$ & $V$ \\ 
                \hline
            \end{tabular}\\[1.0em]

            \subsubsection*{Ideas Importantes}

                La implicación es creo yo la más importante de todas, y no es porque sea básica, 
                es más: $p \to q$ es totalmente equivalente a $\lnot p \lor q$.

                Usando la implicación hay algunas cosas famosas que deberías saber:

                \begin{itemize}
                    \item \textbf{Contrapositiva del Condicional}
                            Esta equivalencia es muy importante, pues es muy usada para las demostraciones
                            (no te preocupes Timmy, ya entenderas después).
                            \begin{equation*}
                                p \to q \lequal \lnot q \to \lnot p
                            \end{equation*}

                    \item \textbf{Implicaciones Famosas}
                            No se a quién se le ocurrio ponerles nombres, pero creo que te combiene
                            que las conozcas.

                            \begin{tabular}{ |c|c|c| } 
                                \hline &&\\
                                \large{Nombre} & \large{Forma} & \large{Es equivalente con...}      \\[0.5em]
                                \hline \hline
                                
                                \textbf{Condicional}    & $p \to q$             & Contrapositiva    \\ \hline
                                \textbf{Contrapositiva} & $\lnot q \to \lnot p$ & Condicional       \\ \hline\hline 
                                
                                \textbf{Recíproca}      & $q \to p$             & Inversa           \\ \hline
                                \textbf{Inversa}        & $\lnot p \to \lnot q$ & Recíproca         \\ \hline
                            \end{tabular}
                \end{itemize}


        % =====================================
        % ======     BICONDICIONAL    =========
        % =====================================
        \clearpage
        \subsection{Bicondicional}

            En lógica la idea de $(p \to q ) \land (q \to p)$ aparece tan seguido que decidimos darle su
            propio símbolo $p \doublearrow q$.

            Esta operación nos regresa verdadero \textbf{solo} cuando ambas premisas tengan el mismo valor de
            verdad. Ojo no dije que ambas sean verdad, simplemente que si una es falsa, obliga a la otra a 
            ser falsa.

            Recuerda que sabemos que $p \to q$ se lee como 'p si q' y $q \to p$ se lee como 'p solo si q'.
            Entonces nuestro nuevo operador recibe el original nombre de 'p si y solo si q' o de forma normal
            'p ssi q'.



    % =====================================================
    % ============           DEFINICION            ========
    % =====================================================
    \section{Equivalente Lógico}

        Llega a pasar en lógica que tenemos dos expresiones lógicas que al momento de ver su tabla de 
        verdad vemos que son iguales en todos los valores de verdad de sus variables entonces podemos
        decir que son logicamente equivalentes. Y solemos denotar eso con este símbolo $p \lequal q$.

        Usamos este símbolo porque si $p$ y $q$ son logicamente equivalentes entonces $p \doublearrow q$
        será siempre verdad, una tautología.

        Esta idea es importante pues nos permite ver ideas que ya tenemos expresadas en la lógica de una
        manera completamente nueva si es que a nosotros nos convienen más.

        A continuación te muestro una tabla con las equivalencias lógicas mas comúnes.






    % =====================================================
    % ============    LEYES DE LOGICA            ==========
    % =====================================================
    \clearpage
    \section{Leyes de Lógica}
                
            Sean $p, q, r$ sentencias lógicas y sea $T$ una tautológía y sea $F$ una contradicción.

            \begin{itemize}
                \item \textbf{Doble Complemento} \\
                        $\lnot(\lnot p) \lequal p$

                \item \textbf{Propiedad Conmutativa}
                    \begin{itemize}
                        \item $p \land q \lequal q \land p$
                        \item $p \lor  q \lequal q \lor  p$
                    \end{itemize}

                \item \textbf{Propiedad Asociativa}
                    \begin{itemize}
                        \item $p \land (q \land r) \lequal (p \land q) \land C$
                        \item $p \lor (q \lor r) \lequal (p \lor q) \lor C$
                    \end{itemize}

                \item \textbf{Propiedad Distributiva}
                    \begin{itemize}
                        \item $p \land (q \lor r) \lequal (p \land q) \lor (p \land r)$
                        \item $p \lor (q \land r) \lequal (p \lor q) \land (p \lor r)$
                    \end{itemize}

                \item \textbf{Leyes de Morgan}
                    \begin{itemize}
                        \item $\lnot (p \land q) \lequal (\lnot p) \lor (\lnot q)$
                        \item $\lnot (p \lor q) \lequal (\lnot p) \land (\lnot q)$
                    \end{itemize}

                \clearpage

                \item \textbf{Propiedad de los Neutros}
                    \begin{itemize}
                        \item $p \land T \lequal p$
                        \item $p \lor F \lequal p$
                    \end{itemize}

                \item \textbf{Propiedad de los Inversos}
                    \begin{itemize}
                        \item $p \land \lnot p \lequal F$
                        \item $p \lor \lnot p \lequal T$
                    \end{itemize}

                \item \textbf{Propiedad de Dominación}
                    \begin{itemize}
                        \item $p \land F \lequal F$
                        \item $p \lor T \lequal T$
                    \end{itemize}

                \item \textbf{Propiedad de Inepotencia}
                    \begin{itemize}
                        \item $p \land p \lequal p$
                        \item $p \lor p \lequal p$
                    \end{itemize}

                \item \textbf{Propiedad de Absorción}
                    \begin{itemize}
                        \item $p \land (p \lor q) \lequal p$
                        \item $p \lor (p \land q) \lequal p$
                    \end{itemize}

                \item \textbf{Propiedad de Contrapositiva}
                    \begin{itemize}
                        \item $p \to q \lequal \lnot q \to \lnot p$
                        \item $p \doublearrow q  \lequal \lnot p \doublearrow \lnot q$
                    \end{itemize}

            \end{itemize}









% ======================================================================================
% ========================              INFERENCIAS             ========================
% ======================================================================================
\clearpage
\chapter{Inferencias Lógicas}
    \clearpage

    
    % =====================================================
    % ============           INFERENCIAS           ========
    % =====================================================
    \section{Inferencias Lógicas}

        La inferencia es la forma en la que obtenemos conclusiones en base a datos y declaraciones establecidas.
        Esto se va a poner intenso, pero creo que esta definición vale la pena:

        \subsubsection{Definición}

            Podemos entonces definir que una inferencia lógica es un proposición $q$ que si le aplicamos
            el condicional con la disyunción de todas las premisas sería una tautología.

            Es decir:
            \begin{equation}
                [p_1 \land p_2 \land p_3 \dots \to q] \lequal T
            \end{equation}

            En español esto quiere decir que el hecho de que todas las premisas sean verdaderas obliga a
            que $q$ sea verdadera, o en otra manera podemos decir que la inferencia lógica como: 
            Dadas dos afirmaciones verdaderas podemos inferir que una tercera afirmación es verdadera.


        \subsubsection{Ejemplo}
            Supongamos que sabemos que una afirmación de la forma $p \to q$ es verdadera.
            Esto nos dice que siempre que $p$ es verdadera, $q$ también será verdadera. 

            Por sí mismo, $p \to q$ siendo verdadero no nos dice que $p$ o $q$ es verdadero
            (ambos podrían ser falsos, o $p$ podría ser falso y $q$ verdadero).

            Sin embargo, si además sabemos que $p$ es verdadera entonces debe ser que $q$ es verdadera.
            Esto se llama una inferencia lógica: dadas dos afirmaciones verdaderas podemos inferir que una
            tercera afirmación es verdadera.


        % =====================================
        % =====     INFERENCIAS BASICAS =======
        % =====================================
        \clearpage
        \subsection{Inferencias Básicas}

            Hay unas inferencias my importantes, sobretodo a la hora de demostrar algo, por eso les deje su propia
            sección:

            \begin{itemize}
                \item
                    \textbf{Contrapositiva de la Inferencia}\\
                    $p \linfire q$ si y solo si $\lnot q \linfire \lnot p$

                \item
                    \textbf{Por Contradicción}\\
                    $p \linfire q$ si y solo si $p \land \lnot q \linfire F$

                \item
                    \textbf{Por Corriento del Condicional}\\
                    $p \linfire q \to r$ si y solo si $p \land q \linfire s$


                \item
                    \textbf{Disyunción}\\
                    Si ya sabemos que $p \linfire q$ entonces sabremos que $p \land r \linfire q \land r$

                \item
                    \textbf{Conjunción}\\
                    Si ya sabemos que $p \linfire q$ entonces sabremos que $p \lor r \linfire q \lor r$

                \item
                    \textbf{Condicional}\\
                    Si ya sabemos que $p \linfire q$ entonces sabremos que $r \to p \linfire r \to q$

                \item
                    \textbf{Transitiva}\\
                    Si ya sabemos que $p \linfire q$ y que $q \linfire r$ entonces sabremos que $p \linfire r$
            \end{itemize}        

           



    % ====================================================
    % ============    LEYES DE INFERENCIAS    ============
    % ====================================================
    \clearpage
    \section{Reglas de Inferencias}

        Hay unas inferencias my importantes, casi casi reglas, se las mostraré a continuación:\\

        \begin{multicols}{2}
            
            \large{\textbf{Modus Poness (PP)}}
                \begin{equation*}
                \begin{split}
                    &p \to q        \\
                    &p              \\
                    \midrule
                    &\therefore q
                \end{split}
                \end{equation*}

            \large{\textbf{Modus Tollens (TT)}}
                \begin{equation*}
                \begin{split}
                    &p \to q        \\
                    &\lnot q        \\
                    \midrule
                    &\therefore \lnot p
                \end{split}
                \end{equation*}

        \end{multicols}

        \bigskip

        \begin{multicols}{2}
            
            \large{\textbf{Silogismo Hipotético}}
                \begin{equation*}
                \begin{split}
                    &p \to q        \\
                    &q \to r        \\
                    \midrule
                    &\therefore p \to r
                \end{split}
                \end{equation*}

            \large{\textbf{Silogismo Disyuntivo}}
                \begin{equation*}
                \begin{split}
                    &p \lor q       \\
                    &\lnot p        \\
                    \midrule
                    &\therefore q
                \end{split}
                \end{equation*}

        \end{multicols}

        \bigskip

        \begin{multicols}{2}
            
            \large{\textbf{Amplificación Disyuntiva}}
                \begin{equation*}
                \begin{split}
                    &p              \\
                    \midrule
                    &\therefore p \lor q
                \end{split}
                \end{equation*}

            \large{\textbf{Simplificación Conjuntiva}}
                \begin{equation*}
                \begin{split}
                    &p \land q      \\
                    \midrule
                    &\therefore p
                \end{split}
                \end{equation*}

        \end{multicols}



        \clearpage

        \large{\textbf{Regla de Conjunción}}
        \begin{equation*}
        \begin{split}
            &p                  \\
            &q                  \\
            \midrule
            &\therefore p \land q
        \end{split}
        \end{equation*}


        \bigskip

        \begin{multicols}{2}
            
            \large{\textbf{Ley del Dilema Constructivo (1)}}
                \begin{equation*}
                \begin{split}
                    &p \to q             \\
                    &r \to s             \\
                    &p \land r           \\
                    \midrule
                    &\therefore q \land r
                \end{split}
                \end{equation*}

            \large{\textbf{Ley del Dilema Constructivo (2)}}
                \begin{equation*}
                \begin{split}
                    &p \to q             \\
                    &r \to s             \\
                    &p \lor r            \\
                    \midrule
                    &\therefore q \lor s
                \end{split}
                \end{equation*}

        \end{multicols}

        \bigskip

        \begin{multicols}{2}
            
            \large{\textbf{Ley del Dilema Destructivo (1)}}
                \begin{equation*}
                \begin{split}
                    &p \to q                \\
                    &r \to s                \\
                    &\lnot q \land \lnot s  \\
                    \midrule
                    &\therefore \lnot p \land \lnot r
                \end{split}
                \end{equation*}

            \large{\textbf{Ley del Dilema Destructivo (2)}}
                \begin{equation*}
                \begin{split}
                    &p \to q                \\
                    &r \to s                \\
                    &\lnot q \lor \lnot s   \\
                    \midrule
                    &\therefore \lnot p \lor \lnot r
                \end{split}
                \end{equation*}

        \end{multicols}




% ======================================================================================
% ========================            CUANTIFICADORES           ========================
% ======================================================================================
\clearpage
\chapter{Cuantificadores Lógicos}
    \clearpage

    % =====================================================
    % ============      CUANTIFICADORES            ========
    % =====================================================
    \clearpage
    \section{Cuantificadores}

        Usar los conectores lógicos nos permiten traducir un teorema matemático en ideas lógicas, pero
        añadiré unos nuevos símbolos que nos permitirán traducir aún más ideas.


        \subsubsection{Sentencias Abiertas y Cuantificadores}
            Los cuantificadores trabajan con sentencias abiertas (o también llamadas funciones lógicas),
            son cosas que son parecidas a las proposiciones, pero no lo son exactamente, son cosas como:

            $p(x)$: $x$ es un número par.

            Puesto que la validez de $p(x)$ depende que número sea $x$, $p(x)$ no es no totalmente cierta ni
            totalmente falsa, por lo tanto no es una proposición.

            Los cuantificadores permiten la construcción de proposiciones a partir de oraciones abiertas,
            bien sea particularizando o generalizando. Así, un cuantificador transforma una oración abierta,
            en una proposición a la cual se le asigna un valor de verdad.

            Es decir, los cuantificadores trabajan con sentencias abiertas, ya que al aplicarles un cuantificador
            se vuelven una proposiciones normales.

        \subsubsection{Cuantificadores Ocultos}
            Ahora llegamos al punto muy importante. En matemáticas, la expresión $p(x) \linfire q(x)$
            se entiende que en realidad hablamos de la oración $\forall x \in A, \MiniSpace p(x) \linfire q(x)$.

            Si, lo se, matemáticos que les da flojera ser formales, aunque entiendelos, es tal común esta
            clase de enunciados que se notaría tan repetetitivo.



        % =====================================
        % ====  CUANTIFICADOR UNIVERSAL   =====
        % =====================================
        \clearpage
        \subsection{Cuantificador Universal}

            Se utiliza para afirmar que \textbf{todos} los elementos de un conjunto $A$ cumplen con una 
            propiedad determinada $p(x)$.

            \begin{equation}
                \forall x \in A,\MiniSpace p(x)
            \end{equation}

            Es normal en matemáticas básicas escuchar frases como $p(a)$ para una $a$ cualquiera, esto es
            simplemente otra forma de decir $\forall x ,\MiniSpace p(x)$.\\

            Otra forma de escribir el cuantificador universal $\forall x \in A,\MiniSpace p(x)$ es
            escribir $(x \in A) \linfire p(x)$



        % =====================================
        % ====  CUANTIFICADOR EXISTENCIAL  ====
        % =====================================
        \subsection{Cuantificador Existencial}

            Se utiliza para afirmar que \textbf{existe al menos un} elemento de un conjunto $A$ que 
            cumple con una propiedad determinada $p(x)$.

            \begin{equation}
                \exists x \in A,\MiniSpace p(x)
            \end{equation}

            Es normal en matemáticas básicas escuchar frases como $p(a)$ para una $a$ específica, esto es
            simplemente otra forma de decir $\exists x, \MiniSpace p(x)$



    % =====================================================
    % =======   LEYES DE CUANTIFICADORES      =============
    % =====================================================
    \clearpage
    \section{Leyes de Cuantificadores}
                
            Sean $p_{(x)}$ sentencias abierta lógica, $A$ un conjunto que opera sobre $p(x)$ donde $x$ son
            elementos de $A$ y sea $T$ una tautología y sea $F$ una contradicción.

            \begin{itemize}
                \item
                \textbf{Negación del Universal} \\
                $\lnot (\forall x \in A,\MiniSpace p(x)) \lequal \exists x \in A,\MiniSpace \lnot p(x)$

                \item
                \textbf{Negación del Existencial} \\
                $\lnot (\exists x \in A,\MiniSpace p(x)) \lequal \forall x \in A,\MiniSpace \lnot p(x)$

                \item \textbf{Cambio de Variables} \\
                        $p(a) \lequal (p(x) \land (x=a))$
                        
                \item \textbf{Cuantificadores sobre Proposiciones} \\
                        $\exists x, p \lequal \forall x, p \lequal p$

                \item \textbf{Leyes Conmutativas para Cuantificador Existencial}

                    \begin{itemize}

                        \item
                        $\exists x, [p(x) \lor q(x)] \lequal \exists x, p(x) \lor \exists x, q(x)$

                        \item
                        $\exists x, [p(x) \land q(x)] \linfire \exists x, p(x) \land \exists x, q(x)$

                    \end{itemize}

                \item \textbf{Leyes Conmutativas para Cuantificador Universal}

                    \begin{itemize}
                        
                        \item
                        $\forall x, [p(x) \land q(x)] \lequal \forall x, p(x) \land \forall x, q(x)$

                        \item
                        $\forall x, p(x) \lor \forall x, q(x) \linfire \forall x, [p(x) \lor q(x)]$

                    \end{itemize}
                    
            \end{itemize}






\end{document}
